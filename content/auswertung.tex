\section{Auswertung}
\label{sec:Auswertung}
\subsection{Wärmekapazität des Kalorimeters}
Um die spezifische Wärmekapazität der Stoffe aus den Messdaten bestimmen
zu können wird zunächst die Wärmekapazität des Kalorimeters benötigt,
in welchem die Messungen durchgeführt wurden. Hierzu wurde \ref{sec:D1}
durchgeführt und
durch einsetzen der Messwerte in \eqref{eqn:cgmgumf} ergibt sich für die Wärmekapazität des
Kalorimeters
\begin{equation*}
  c_\symup{g}m_\symup{g} = \SI{202,79}{\joule\per\kelvin}
\end{equation*}

\subsection{spezifische Wärmekapazität der einzelnen Metalle}
Zunächst werden die Mittelwerte der jeweiligen Messungen für die
einzelne Metalle nach \eqref{eqn:mittel} sowie deren Fehler nach \eqref{eqn:err}, unter Hilfenahme von
Python, berechnet.
Unter Berücksichtigung von \eqref{eqn:gauß} ergibt sich durch einsetzen
in \eqref{eqn:ckauswertung}
\begin{gather*}
  \intertext{für Aluminium:}
  c_\symup{k} =\SI{0,52(15)}{\joule\per\gram\per\kelvin}\\
  \intertext{für Zinn:}
  c_\symup{k} =\SI{0,20(9)}{\joule\per\gram\per\kelvin}\\
  \intertext{für Blei:}
  c_\symup{k} =\SI{0,26(14)}{\joule\per\gram\per\kelvin}
\end{gather*}

\subsection{Atomwärme im Vergleich zum Vorhersagewert 3R}
Im folgenden soll nun bestimmt werden, wie genau der Dulong-Petitsche
Wert $c_\symup{V}\approx$ 3R $\approx$ \SI{24,942}{\joule\per\mol\per\kelvin}
für den Realfall, sprich die Messung, ist. Die gemessene Atomwärme
berechnet sich nach \eqref{eqn:CVumf} unter Berücksichtigung von
$C_\symup{p}=Mc_\symup{k}$, $V_0 = M/\rho$ und $T=T_\text{gemischt}$
\begin{equation}
  C_\symup{V}=Mc_\symup{k}-9\alpha^2\kappa\frac{M}{\rho}T_\text{gemischt}
  \label{eqn:CV}
\end{equation}
Die Werte $\alpha$, $\kappa$, $\rho$ und $M$ sind dabei der
Anleitung zu entnehmen. Weiterhin unter Berücksichtigung von \eqref{eqn:gauß}
ergeben sich nun für die einzelnen Metalle die Atomwärmen und deren
prozentuale Abweichung von der Dulong-Petitschen Konstante
$c_\symup{V}\approx$ \SI{24,942}{\joule\per\mol\per\kelvin}
\begin{align*}
  \intertext{für Aluminium:}
  C_V&=\SI{14(4)}{\joule\per\mol\per\kelvin}&
  \text{Abweichung}&=\SI{43(16)}{\percent}\\
  \intertext{für Zinn:}
  C_V&=\SI{24(10)}{\joule\per\mol\per\kelvin}&
  \text{Abweichung}&=\SI{10(40)}{\percent}\\
  \intertext{für Blei:}
  C_V&=\SI{53(29)}{\joule\per\mol\per\kelvin}&
  \text{Abweichung}&=\SI{110(120)}{\percent}
\end{align*}
