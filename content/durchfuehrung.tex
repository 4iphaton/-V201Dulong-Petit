\section{Aufbau und Durchführung}
\label{sec:Durchführung}
Das Experiment besteht aus 2 Teilen. Im ersten Teil \ref{sec:D1} soll die Wärmekapazität
des Kalorimeters (Abbildung 2) $c_\symup{g}m_\symup{g}$ bestimmt werden um im 2.
Teil \ref{sec:D2} durch \eqref{eqn:ckauswertung} und \eqref{eqn:CVumf} die
spezifische Wärmekapazität $c_\symup{k}$ und den Vergleichswert $\symup{C}_\symup{V}$
berechnen zu können.
\subsection{Bestimmung der Wärmekapazität des Kalorimeters}
\label{sec:D1}
Zunächst werden zwei verschiedene Mengen Wasser unterschiedlicher Temperatur
im Kalorimeter zusammen gemischt.
Es gilt, dass die Wärmekapazität des warmen Wassers
$c_\text{wasser}m_\text{warm}$ mal der Temperaturdifferenz des warmen Wassers
$T_\text{warm}$ und der Mischtemperatur $T_\text{gemischt}$ gleich der
Temperaturdifferenz des kalten Wassers $T_\text{kalt}$ und $T_\text{gemischt}$
mal der Wärmekapazität des kalten Wassers $c_\text{wasser}m_\text{kalt}$
und der des Kalorimeters $c_\symup{g}m_\symup{g}$ ist.
\begin{gather}
  \left(c_\text{wasser}m_\text{kalt}+c_\symup{g}m_\symup{g}\right)
  \left(T_\text{gemischt}-T_\text{kalt}\right)
  = c_\text{wasser}m_\text{warm}\left(T_\text{warm}-T_\text{gemischt}\right)
  \label{eqn:cgmg}\\
  \intertext{Wodurch sich folgendes für $c_\symup{g}m_\symup{g}$ ergibt}
  c_\symup{g}m_\symup{g} = \frac
  {c_\text{wasser}m_\text{warm}\left(T_\text{warm}-T_\text{gemischt}\right)
  -c_\text{wasser}m_\text{kalt}\left(T_\text{gemischt}-T_\text{kalt}\right)}
  {\left(T_\text{gemischt}-T_\text{kalt}\right)}
  \label{eqn:cgmgumf}
\end{gather}
\subsection{Bestimmung der spezifischen Wärmekapazität der Metalle}
\label{sec:D2}
Im 2. Teil des Versuches sollen die spezifischen Wärmekapazitäten von
den Metallen Aluminium, Zinn und Blei bestimmt werden. Hierzu werden die Metalle
zunächst in einem Wasserbad erhitzt (Abbildung 2) und im Kalorimeter in einem
kalten Wasserbad abgekühlt wobei die Temperatur des warmen Metalls wie sein Gewicht,
die Temperatur und das Gewicht des kalten Wassers im Kalorimeter
und die finale Temperatur des Wassers nach dem Wärmeaustausch notiert werden.
Dies liefert sämtliche Daten zur Berechnung der spezifischen Wärmekapazität
$c_\symup{k}$ unter anwendung von \eqref{eqn:ckauswertung}.
