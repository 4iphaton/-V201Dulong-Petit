\section{Theorie}
\label{sec:Theorie}
Die Mittelwerte und Fehler der Messungen werden über Python berechnet nach
\begin{gather}
  \left<x\right> = \frac{1}{N}\sum_{i=1}^N x_i
  \label{eqn:mittel}\\
  \sigma_x = \sqrt{\frac{1}{N}\sum_{i=1}^N \left(\left<x\right>-x_i\right)^2}
  \label{eqn:err}\\
  \increment f = \sqrt{\sum_{j=1}^k\left(\frac{\partial f\left(x_j\right)}{\partial x_j}\cdot\sigma_j\right)^{\!\!2}}
  \label{eqn:gauß}
\end{gather}
\subsection{Definition: spezifische Wärmekapazität}
Unter der Bedingung, dass keine Arbeit verrichtet wird, steht die spezifische
wärmekapazität im direkten Zusammenhang mit der Wärmemenge $\increment Q$ und
der Temperatur $\increment T$
\begin{equation}
  \increment Q = mc_\symup{w}\increment T
\end{equation}
Die sogenannte Molwärme $C$ beschreibt die Kapazität eines Stoffes bezüglich
Wärmeauf- und abnahme. Dies wird definiert durch eine bestimmte Wärmemenge
d$Q$ pro Temperatur d$T$ bei konstantem Volumen V.
\begin{equation}
  C_\symup{V} = \left(\frac{\symup{d}Q}{\symup{d}T}\right)_{\!\!\symup{V}}
\end{equation}
Dies lässt sich nach dem 1. Hauptsatz der Thermodynamik umformen zu
\begin{equation}
  C_\symup{V} = \left(\frac{\symup{d}U}{\symup{d}T}\right)_{\!\!\symup{V}}
  \label{eqn:CVdif}
\end{equation}
Konkret ergibt sich der Zusammenhang
\begin{equation}
    C_\symup{V}=C_\symup{p}-9\alpha^2\kappa V_0T
    \label{eqn:CVumf}
\end{equation}
Definiert ist die Molwärme $C$ als das Produkt von Masse $m$ und spezifischer
Wärmekapazität $c_\symup{w}$
\begin{equation}
  C = mc_\symup{w}
\end{equation}
Über das Experiment lässt sich die spezifische Wärmekapazität eines Metalls
nach folgender Formel berechnen
\begin{equation}
  c_\symup{k} = \frac{\left(c_\text{wasser}m_\text{wasser}+c_\symup{g}m_\symup{g}\right)
\left(T_\text{gemischt}-T_\text{wasser}\right)}{m_\text{metal}\left(T_\text{metal}-T_\text{gemischt}\right)}
\label{eqn:ckauswertung}
\end{equation}
Hierbei ist $c_\symup{g}m_\symup{g}$ die Wärmekapazität des genutzten
Behältnisses
\subsection{Bergündung von Dulong-Petit in der klassischen Mechanik}
Das Dulong-Petitsche Gesetz besagt, dass feste Stoffe bei hohen Temperaturen
stets die Atomwärme
\begin{equation}
  C_\symup{V}= 3\symup{R}
\end{equation}
mit der allgemeinen Gaskonstante R = \SI{8,314}{\joule\per\mol\per\kelvin}
haben. Dies lässt sich durch die energetische Betrachtung der Schwingungen innerhalb
der Gitterstruktur auf dem Weg der klassischen Mechanik herleiten.
In einem großen Zeitraum gemittelt ergibt sich für die innere Energie pro Teilchen
\begin{equation}
  \left<u\right> = \left<E_\text{kin}\right> + \left<E_\text{pot}\right>
  = 2\left<E_\text{kin}\right>
  \label{eqn:E}
\end{equation}
Hierbei ist zu beachten, dass sich für die Energien der Schwingungen der gleiche
Betrag für $\left<E_\text{kin}\right>$ und $\left<E_\text{pot}\right>$ ergibt.
Das Äquipartitionstheorem besagt, dass die mittlere kinetische Energie eines Atoms
pro Freiheitsgrad
$\frac{1}{2}\symup{k}T$ beträgt. Da für die Schwingung drei Freiheitsgrade bestehen
ergibt sich nach \eqref{eqn:E}
\begin{equation}
  \left<u\right> = 3\symup{k}T
\end{equation}
Bei Betrachtung eines Körpers mit $\symup{N}_\symup{L}$ Atomen pro mol, ergibt sich für
seine gemittelte innere Energie unter Berücksichtigung von $\symup{N}_\symup{L}\symup{k} = \symup{R}$
\begin{gather}
  \left<U\right> = 3\symup{R}T\\
  \intertext{nach \eqref{eqn:CVdif} ergibt sich}
  C_\symup{V} = \frac{\left<U\right>}{T} = 3\symup{R}
\end{gather}
\subsection{Quantentheoretischer Ansatz}
Nach experimentellem Befund zeigt sich, dass das Dulong-Petitsche Gesetz
nur bei hohen Temperaturen gilt. Dieser Befund lässt sich nicht durch
einen klassisch mechanischen Ansatz lösen und benötigt daher einen
Ansatz der Quantentheorie. Nach diesem Ansatz kann sich die Gesamtenergie
nur quantisiert ändern und es gilt für einen Körper der mit der Frequenz
$\omega$ schwingt
\begin{equation}
  \increment u = \symup{n}\symup{\hbar}\omega
\end{equation}
aus diesem Ansatz ergibt sich der Ausdruck für die mittlere innere Energie
$\left<U_\text{qu}\right>$
\begin{equation}
  \left<U_\text{qu}\right> = 3\symup{N}_\symup{L}\left<u_\text{qu}\right>
  = \frac{3\symup{N}_\symup{L}\symup{\hbar}\omega}
  {\exp\left(\frac{\symup{\hbar}\omega}{\symup{k}T}\right)-1}
\end{equation}
Nach einer Taylorentwicklung ergibt sich im Grenzwert für $T$ gegen
\infty
\begin{equation}
  \left<U_\text{qu}\right> = 3\symup{R}T
\end{equation}
Hierdurch ist gezeigt, dass der Dulong-Petitsche Wert $C_\symup{V}=3\symup{R}$
auch nach dem quantentheoretischen Ansatz eine gute Näherung für
hohe Temperaturen ist.
