\section{Theorie}
\label{sec:Theorie}
\subsection{Definition: spezifische Wärmekapazität}
Unter der Bedingung, dass keine Arbeit verrichtet wird, steht die spezifische
wärmekapazität im direkten Zusammenhang mit der Wärmemenge $\increment Q$ und
der Temperatur $\increment T$
\begin{equation}
  \increment Q = mc\increment T
\end{equation}
Die sogenannte Molwärme $C$ beschreibt die Kapazität eines Stoffes bezüglich
Wärmeauf- und abnahme. Dies wird definiert durch eine bestimmte Wärmemenge
d$Q$ pro d$T$.
\begin{equation}
  C_\symup{V} = \left(\frac{\symup{d}Q}{\symup{d}T}\right)_{\!\!\symup{V}}
\end{equation}
\subsection{Bergündung von Dulong-Petit in der klassischen Mechanik}
\subsection{Quantentheoretischer Ansatz}
