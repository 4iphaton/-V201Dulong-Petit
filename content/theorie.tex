\section{Theorie}
\label{sec:Theorie}
\subsection{Definition: spezifische Wärmekapazität}
Unter der Bedingung, dass keine Arbeit verrichtet wird, steht die spezifische
wärmekapazität im direkten Zusammenhang mit der Wärmemenge $\increment Q$ und
der Temperatur $\increment T$
\begin{equation}
  \increment Q = mc_\symup{w}\increment T
\end{equation}
Die sogenannte Molwärme $C$ beschreibt die Kapazität eines Stoffes bezüglich
Wärmeauf- und abnahme. Dies wird definiert durch eine bestimmte Wärmemenge
d$Q$ pro Temperatur d$T$.
\begin{equation}
  C_\symup{V} = \left(\frac{\symup{d}Q}{\symup{d}T}\right)_{\!\!\symup{V}}
\end{equation}
Dies lässt sich nach dem 1. Hauptsatz der Thermodynamik umformen zu
\begin{equation}
  C_\symup{V} = \left(\frac{\symup{d}U}{\symup{d}T}\right)_{\!\!\symup{V}}
  \label{eqn:CVdif}
\end{equation}
Definiert ist die Molwärme $C$ als das Produkt von Masse $m$ und spezifischer
Waermekapazität $c$
\begin{equation}
  C = mc_\symup{w}
\end{equation}

\subsection{Bergündung von Dulong-Petit in der klassischen Mechanik}
Das Dulong-Petitsche Gesetz besagt, dass feste Stoffe bei hohen Temperaturen
stets die Atomwärme
\begin{equation}
  C_\symup{V}= 3\symup{R}
\end{equation}
mit der allgemeinen Gaskonstante R = \SI{8,314}{\joule\per\mol\per\kelvin}
haben. Dies lässt sich durch die energetische Betrachtung der Schwingungen innerhalb
der Gitterstruktur auf dem Weg der klassischen Mechanik herleiten.
In einem großen Zeitraum gemittelt ergibt sich für die innere Energie pro Teilchen
\begin{equation}
  \left<u\right> = \left<E_\text{kin}\right> + \left<E_\text{pot}\right>
  = 2\left<E_\text{kin}\right>
  \label{eqn:E}
\end{equation}
Hierbei ist zu beachten, dass sich für die Energien der Schwingungen der gleiche
Betrag für $\left<E_\text{kin}\right>$ und $\left<E_\text{pot}\right>$ ergibt.
Das Äquipartitionstheorem besagt, dass die mittlere kinetische Energie eines Atoms
pro Freiheitsgrad
$\frac{1}{2}\symup{k}T$ beträgt. Da für die Schwingung drei Freiheitsgrade bestehen
ergibt sich nach \eqref{eqn:E}
\begin{equation}
  \left<u\right> = 3\symup{k}T
\end{equation}
Bei Betrachtung eines Körpers mit $\symup{N}_\symup{L}$ Atomen pro mol, ergibt sich für
seine gemittelte innere Energie unter Berücksichtigung von $\symup{N}_\symup{L}\symup{k} = \symup{R}$
\begin{gather}
  \left<U\right> = 3\symup{R}T\\
  \intertext{nach \eqref{eqn:CVdif} ergibt sich}
  C_\symup{V} = \frac{\left<U\right>}{T} = 3\symup{R}
\end{gather}
\subsection{Quantentheoretischer Ansatz}
