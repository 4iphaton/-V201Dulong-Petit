\section{Diskussion}
\label{sec:Diskussion}
Die Ergebnisse der Messungen für Zinn und Blei umfassen mit
ihrem zugehörigen Fehler den Wert von 3R die Messungen für Aluminium unterscheiden
sich jedoch ziemlich stark. Dies ist zum einen durch Messfehler zu begründen,
da der bereits berechnete Wert für $c_\symup{k}$ von Aluminium $c_{\symup{k}\text{, Aluminium}} =$
 \SI{0,52(15)}{\joule\per\gram\per\kelvin} die größte Abweichung
gegenüber dem Literaturwert $c_{\symup{k}\text{, Aluminium}}$ = \SI{0,897}
{\joule\per\gram\per\kelvin} zeigt. Aluminium gehört jedoch auch zu den Leichtmetallen,
sprich es hat ein geringes Atomgewicht, bei welchen der Grenzfall des Dulong-Petitschen
Wertes erst bei \SI{1000}{\celsius} erreicht wird.
Eine weitere Abweichungen kommt dadurch zustande, dass sich die Wärme nicht gleichmäßig im Metall verteilt hat,
sowie der Tatsache, dass Wärme aus dem System nach außen dringen
kann, da dieses nicht sonderlich gut isoliert ist und das Metall zwischen
der Entnahme aus dem Wasserbad und dem Eintauchen in das Kalorimeter
bereits kurze Zeit an der Luft abkühlen kann. Des weiteren,
auch wenn die eigentlichen Werte von $c_\symup{k}$ zumindest für
Zinn und Blei den Literaturwerten\cite{chemie}  $c_{k\text{, Zinn}}$ = \SI{0,23}{\joule\per\gram\per\kelvin}
und $c_{k\text{, Blei}}$ = \SI{0,129}{\joule\per\gram\per\kelvin}
sehr ähnlich sind, sind deren Fehler doch
sehr groß was ein weiterer Grund für die Abweichung der Messwerte
von 3R ist. Abschließend lässt sich sagen das die Dulong-Petitsche
Konstante eine gute grobe Näherung sein kann, wenn man Metalle mit einer
hohen Wärmeleitfähigkeit bei hohen Temperaturen benutzt.
