\section{Diskussion}
\label{sec:Diskussion}
Die Ergebnisse der Messungen für Aluminium und Zinn umfassen mit
ihrem zugehörigen Fehler den Wert von 3R die Messung für Blei jedoch nicht.
Dies liegt wahrscheinlich daran das Blei von allen 3 Metallen die
Wärme des Wasserbads, aufgrund seiner geringen Wärmeleitfähigkeit,
am langsamsten aufnimmt und somit am wenigsten
vollständig erhitzt ist. Auch bei den anderen Metallen kommen große
Abweichungen zustande, welche auch hier zum Teil darauf zurück zu führen
sind, dass sich die Wärme nicht gleichmäßig im Metall verteilt hat.
Eine weitere Fehlerquelle ist, dass Wärme aus dem System nach außen dringen
kann, da dieses nicht sonderlich gut isoliert war und das Metall zwischen
der Entnahme aus dem Wasserbad und dem Eintauchen in das Kalorimeter
bereits kurze Zeit an der Luft abkühlen konnte. Des weiteren,
auch wenn die eigentlichen Werte von $c_\symup{k}$ zumindest für
Zinn und Aluminium den Literaturwerten\cite{chemie} $c_{k\text{, Aluminium}}$ =\SI{0,897}
{\joule\per\gram\per\kelvin} und $c_{k\text{, Zinn}}$ =\SI{0,23}
{\joule\per\gram\per\kelvin} sehr ähnlich sind, sind deren Fehler doch
sehr groß was ein weiterer Grund für die Abweichung der Messwerte
von 3R ist. Abschließend lässt sich sagen das die Dulong-Petitsche
Konstante eine gute grobe Näherung sein kann, wenn man Metalle mit einer
hohen Wärmeleitfähigkeit bei hohen Temperaturen benutzt.
